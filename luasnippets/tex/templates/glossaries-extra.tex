% Use glossaries{,-extra} over e.g. acro since the former is more established and is recompiled separately.
% This means compilation should be faster since we can separate glossary compilation (with leader-leader-g) and text compl.
% toc = add glossary to table of contents.
% postdot = add a dot/period after each entry in the printing of the glossary. 
% We do this since we also have option for a text description and page number 
% of term usage.
% We use -extra since it adds categories so we can have different print styles and different abbreviation styles per category.
% shortcuts=ac means we can write \ac, \acp, \Aclp instead of \cgls, \cglspl, \Glsxtrlongpl. \cgls as opposed to \gls means ability to use optional entry counting that we don't currently use.
% xindy, which replaces the default makeindex. xindy ignores latex commands in sorting, which means I don't have to add a sort key for \textit{name} entries, which would be sorted by makeindex by "\".
\usepackage[toc,postdot,shortcuts=ac,xindy]{glossaries-extra}
% Disable links from terms as they are used to the glossary but not the other way.
\setkeys{glslink}{hyper=false}

% listgroup adds "A" above all entries starting with "A", etc.
\setglossarystyle{listgroup}
% For no good reason they decided to place a \penalty (for pagebreak) after groupings for the listgroup, which means after the "A" before all entries starting with "A". I redefine the skip to its default excluding the penalty. Requires the styling package.
\usepackage{glossaries-extra-stylemods}
\renewcommand*{\glsgroupskip}{\indexspace}
% \renewcommand*{\glslistgroupskip}{\indexspace} % Can also be done for a specific style only.

% the -extra package has options for categories acronym, etc. that can be 
% treated separately, with different abbreviationstyles and can be printed to 
% separate groupings.
% We only want them to have separate styles for how they are displayed on first 
% usage so we only take advantage of the category concept and define our own:
\setabbreviationstyle[acronym]{short-long}
\setabbreviationstyle[acronymdesc]{short-long-desc}
\setabbreviationstyle[abbreviationdesc]{long-short-desc}
% Then we make shorthand commands for each:
\renewcommand{\newacronym}[3]{\newabbreviation[category=acronym]{#1}{#2}{#3}}
\newcommand{\newacronymdesc}[4]{\newabbreviation[category=acronymdesc,description={#4}]{#1}{#2}{#3}}
\newcommand{\newabbreviationdesc}[4]{\newabbreviation[category=abbreviationdesc,description={#4}]{#1}{#2}{#3}}
% There gives us 5 options for making an entry:
% \newabbreviation using default long-short
% \newabbreviationdesc using long-short-desc
% \newacronym using short-long
% \newacronymdesc using short-long-desc
% \newglossaryentry using name-desc aka short-nolong-desc (?)

% The long-short-desc has a different print style than long-short, i.e. instead of <SHORT> <LONG> it prints <LONG> (<SHORT>) <DESC>.
% We change that here so they are consistent: <SHORT> <LONG> <DESC>
% I found the default sorting and naming for it in the glossaries-extra manual as: \glsxtrlongshortdescsort and \glsxtrlongshortdescname
% Then just modified them to swap short and long and unbold, remove parenthesis, add a dot.
\renewcommand{\glsxtrlongshortdescsort}{\expandonce\glsxtrorgshort\space \expandonce\glsxtrorglong}
\renewcommand{\glsxtrshortlongdescsort}{\expandonce\glsxtrorgshort\space \expandonce\glsxtrorglong}
\renewcommand{\glsxtrlongshortdescname}{
\glsxpabbrvfont{\the\glsshorttok}{\glscategorylabel}%
\protect\glsxtrfullsep{\the\glslabeltok}%
\normalfont % Don't make it bold
\protect
\glsxplongfont{\the\glslongtok}{\glscategorylabel}. % notice a period is added.
}
\renewcommand{\glsxtrshortlongdescname}{
\glsxpabbrvfont{\the\glsshorttok}{\glscategorylabel}%
\protect\glsxtrfullsep{\the\glslabeltok}%
\normalfont % Don't make it bold
\protect
\glsxplongfont{\the\glslongtok}{\glscategorylabel}. % notice a period is added.
}

% The following creates a cmd \itag that can be used in the "long" to underline the letters that are used in an acronym.
\GlsXtrEnableInitialTagging{acronym,abbreviation,acronymdesc,abbreviationdesc}{\itag}

\makeglossaries
\loadglsentries{glossary}
% Use this instead for subfiles package:
% \loadglsentries{\subfix{glossary}}

