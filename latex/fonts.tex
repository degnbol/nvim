% !TEX program = LuaLaTeX
\documentclass[a4paper,10pt]{article}
\usepackage[margin=2cm, top=0.5in]{geometry}
\usepackage{xspace} % \xspace at end of newcommand allows "\CUSTOM " instead of "\CUSTOM\ "

% All the fonts are rendering fine without babel
% \usepackage[greek,danish,english]{babel}
\usepackage{csquotes}
\usepackage{fontspec}

% Improve default latex packages.
% allowing (2%) letter stretch
\usepackage{microtype}
% \raggedright -> \RaggedRight, \flushleft env -> \FlushLeft, \center -> \Center
% https://www.overleaf.com/learn/latex/Text_alignment
\usepackage{ragged2e}
% \usepackage{flushend} % enable for two columns, to get equal lengths on last page.

% \begin{itemize}[label={--}] instead of repeating /item[--]
\usepackage{enumitem}
\usepackage{fancyhdr} % header/footer control
% paragraph space instead of indent
\usepackage[parfill]{parskip} 

% math
\usepackage{mathtools} % loads amsmath, plus e.g. \coloneqq,\mathclap,\substack
\usepackage{amssymb}
\usepackage{siunitx} % provides \SI and column type S (align decimal)
\usepackage{unicode-math} % experimental support for unicode in math

\usepackage{graphicx}
% Subfigures. "skip" == spacing between subfigures.
\usepackage[skip=0pt]{subcaption}
\usepackage{float}
% makes figures and tables stay in their section
\usepackage[section]{placeins}

% Tables.
% https://tex.stackexchange.com/questions/12672/which-tabular-packages-do-which-tasks-and-which-packages-conflict
\usepackage{array} % flexible column formatting
\usepackage{tabularx} % Column type X for width filling.
\usepackage{tabulary} % Column type L,C,R,J for balanced width versions of l,c,r,j.
\usepackage{booktabs} % Better vertical spacing. Midrule etc with varying thickness instead of \hline.
% creates missing tabularx column type named "R" to specify right adjustment
\newcolumntype{R}{>{\raggedleft\arraybackslash}X}
\usepackage{multicol, multirow} % Also see pbox.
% \usepackage{longtable} % Multipage table. Might not support column X. Alts: xltabular, ltxtable

\usepackage{xcolor} % define colors
\usepackage{hyperref}
% \usepackage[capitalise]{cleverref} % \cref which auto adds e.g. "Table " to \ref
% Auto define acronyms on first use. https://www.overleaf.com/learn/latex/Glossaries
\usepackage[acronym]{glossaries-extra}
\setabbreviationstyle[acronym]{long-short}
% bibliography
\usepackage{biblatex}
% appendix
\usepackage[toc,page]{appendix}
% Code blocks.
% https://www.overleaf.com/learn/latex/Code_listing
% https://www.overleaf.com/learn/latex/Code_Highlighting_with_minted
% \usepackage{listings}
% \usepackage{minted} % supports julia

\begin{document}

% Comparing default font to similar fonts.

% https://en.wikipedia.org/wiki/Computer_Modern
% The default choice should be New Computer Modern 10.
% mlmodern is not standard, i.e. maybe not available.
% Default doesn't have unicode characters included, and people argue it is 
% uglier than lmodern:
% https://tex.stackexchange.com/questions/1390/latin-modern-vs-cm-super/65103#65103
% lmodern (latin modern roman or \usepackage{lmodern}) doesn't have unicode 
% included. There is also a math version of lmodern, which does have greek, but 
% not ligatures, see the "fi". It's greek also doesn't have the right look. New 
% Computer Modern 10 is a nicer reworked version with the pretty ß like 
%lmodern, does have unicode and has slightly thicker lines which is better on 
%screen.

									 fis α Méièrstraße. Øens å kan være blå.\\
\setmainfont{latin modern roman}     fis α Méièrstraße. Øens å kan være blå.\\
\setmainfont{latin modern math}      fis α Méièrstraße. Øens å kan være blå.\\
\setmainfont{CMUSerif}               fis α Méièrstraße. Øens å kan være blå.\\
\setmainfont{New Computer Modern 10} fis α Méièrstraße. Øens å kan være blå.\\

\end{document}
